\documentclass[11pt]{book}

\usepackage{amsmath}
\usepackage{amssymb}
\usepackage{amsxtra}
\usepackage{wasysym}
\usepackage{array}
\usepackage{wrapfig}
\usepackage[dvips]{epsfig,psfrag}
\usepackage{graphicx}
\usepackage{listings}
\usepackage{tikz}
\usepackage[german]{babel}
\usepackage{color}
\usepackage{palatino}

\newcommand{\refchapter}[1]{Kapitel~\ref{#1}}
\newcommand{\refsec}[1]{Abschnitt~\ref{#1}}
\newcommand{\reffig}[1]{Abbildung~\ref{#1}}
\newcommand{\refex}[1]{Beispiel~\ref{#1}}

\newcommand{\Pasta}{\textcolor{brown}{{\bf \em P}{\scriptsize{(asta)}}}}


\title{\bf Die Programmiersprache \Pasta} 

\author{Prof. Dr. Uwe Naumann}
\date{
LuFG Informatik 12 \\
RWTH Aachen
}

\begin{document}

\pagestyle{plain}

\maketitle

\chapter*{Vorwort}

In dieser Vorlesung stelle ich die (hoffentlich) intuitive Programmiersprache
\Pasta\ vor. Habe sogar einen Compiler (bzw. Interpreter) daf\"ur 
geschrieben! 
%Diesen findet ihr auf
%\color{blue}
%\begin{center}
%\tt http://stce.rwth-aachen.de/Pasta
%\end{center}
%\color{black}
%inklusive einer kurzen Anleitung zu dessen Installation.

Was auch immer in \Pasta\ geht, kann (und werde) ich live
mithilfe eines Stapels von Tellern und einer 
T\"ute voller Nudeln
zeigen. Nachdem ich diverse Nudelumschichtungsdurchg\"ange, z.B.
zur Berechnung der Fakult\"at einer gegeben Zahl (von Nudeln), hinter mich
gebracht haben werde, wird sich der Nutzen der Programmiersprache \Pasta\
eindrucksvoll best\"atigt haben. 
Anschlie\ss end sind weitere Beispiele geplant sowie Details zur 
Implementierung von \Pasta\ in Form des Pasta-Compilers 
\textcolor{red}{\tt Pc}. Gern k\"onnen wir auch gemeinsam an der 
L\"osung eines beliebigen Problems arbeiten oder aber einfach die
Nudeln essen...

Die Programmiersprache \Pasta\ ist in der Entwicklung. Version 1 soll nicht
der Weisheit letzter Schluss sein. Einerseits sind
weitere Beispiele herzlich willkommen! Sendet sie bitte
inklusive 
\begin{enumerate}
\item einer kurzen Beschreibung der zu l\"osenden Aufgabe
\item eines m\"oglichst detailliert dokumentieren Quellcodes
\item einer (oder bei Bedarf auch mehrerer) dokumentierten Beispielsitzung
\end{enumerate}
an
\color{blue}
{\tt naumann@stce.rwth-aachen.de}
\color{black}
... vielen Dank!
Andererseits bin ich auch f\"ur Ideen zur Weiterentwicklung von
\Pasta\ dankbar. Dabei soll vor allem die Nat\"urlichsprachigkeit von
\Pasta\ sichergestellt werden, was gar nicht so einfach ist, wie es 
klingt ... \\
\\
Viel Spa\ss \\
\\
Uwe Naumann
\vfill

\tableofcontents

\lstset{basicstyle=\normalsize,frame=none}

\chapter{Sprache}

\Pasta\ Programme beginnen mit der Bereitstellung von
Speicherplatz in Form von Tellern (siehe \refsec{sec:vars})
und werden durch
\color{red}
\begin{lstlisting}
Bis bald!
\end{lstlisting}
\color{black}
beendet, worauf sich das Programm mittels 
\color{purple}
\begin{lstlisting}
Guten Appetit!
\end{lstlisting}
\color{black}
verabschiedet.

\section{Variablen und Werte} \label{sec:vars}

Der Hauptspeicher des Computers wird durch Teller repr\"asentiert. 
Dessen Gr\"o\ss e wird mithilfe des
\textcolor{red}{\lstinline{[Hole dir ... Teller]}}-Sprachkonstrukts
explizit festgelegt: 
\color{red}
\begin{lstlisting}
Hole dir <wieviele> Teller!
\end{lstlisting}
\color{black}
Individuelle Anweisungen werden mit Ausrufezeichen abgeschlossen.
So wird z.B. mittels
\color{blue}
\begin{lstlisting}
Hole dir 3 Teller!
\end{lstlisting}
\color{black}
Speicher der Gr\"o\ss e 3 reserviert (alloziert). 

Auf den Tellern liegende Nudeln stellen (ganzzahlige) Werte dar.
Die aktuelle Version von \Pasta\ kennt nur ganze Zahlen.
Nach ihrer Allokation sind alle Teller leer.

\paragraph{Dasselbe in C++}
In C++ k\"onnte man ein Feld der Gr\"o\ss e 3 von Elementen ganzzahligen Typs allozieren und zu Null initialisieren.
\begin{lstlisting}
int t[3]={0,0,0};
\end{lstlisting}

\section{Zuweisungen}

Das Hinzuf\"ugen von Nudeln zu Tellern geschieht durch Verwendung des
\textcolor{red}{\lstinline{[Lege ... auf]}}-Sprachkonstrukts:
\color{red}
\begin{lstlisting}
Lege <wieviele> Nudeln auf <welchen Teller>!
\end{lstlisting}
\color{black}
Analog, erlaubt das
\textcolor{red}{\lstinline{[Nimm ... von]}}-Sprachkonstrukt
das Entfernen von Nudeln:
\color{red}
\begin{lstlisting}
Nimm <wieviele> Nudeln von <welchem Teller>!
\end{lstlisting}
\color{black}

Der Zugriff auf die einzelnen Teller geschieht \"uber eine eindeutige
Nummer $1,2,\ldots$, z.B.  \lstinline{Teller 1}, \lstinline{Teller 2}, und
\lstinline{Teller 3}. Alle Zuweisungen sind inkrementell bzw. dekrementell.
So f\"ugt z.B. die Anweisung
\color{blue}
\begin{lstlisting}
Lege 9 Nudeln auf Teller 1!
\end{lstlisting}
\color{black}
neun zu den bereits auf Teller 1 liegenden Nudeln hinzu. Mittels
\color{blue}
\begin{lstlisting}
Nimm 4 Nudeln von Teller 1!
\end{lstlisting}
\color{black}
werden vier Nudeln von Teller 1 entfernt.

\paragraph{Dasselbe in C++}
In C++ k\"onnte man mittels
\begin{lstlisting}
t[0]+=9;
\end{lstlisting}
neun Nudeln auf Teller 1 legen bzw. mittels
\begin{lstlisting}
t[0]-=4;
\end{lstlisting}
davon vier wieder entfernen.




\section{Ein- und Ausgabe}
Zur Ein- bzw. Ausgabe stellt \Pasta\
\textcolor{red}{\lstinline{[Frag mich]}}- bzw.  
\textcolor{red}{\lstinline{[Sag mir]}}-Sprach-konstrukte zur Verf\"ugung.

Die Anzahl der auf einem Teller liegenden Nudeln kann durch Benutzereingabe
explizit gesetzt werden, z.B. resultiert
\color{blue}
\begin{lstlisting}
Frag mich, wieviele Nudeln auf Teller 1 liegen sollen!
\end{lstlisting}
\color{black}
in der Ausgabe
\color{purple}
\begin{lstlisting}
Wieviele Nudeln liegen auf Teller 1?
:-) 
\end{lstlisting}
\color{black}
Von den Benutzer(innen) des Programms wird daraufhin die Eingabe einer ganzen Zahl,
z.B. 3, erwartet. Infolgedessen wird die Anzahl der Nudeln auf Teller 1 auf drei
gesetzt, was einer nichtinkrementellen Zuweisung von drei (Nudeln) zu Teller 1
entspricht.

M\"ochte man wissen, wieviel Nudeln auf einem Teller liegen, so kann man das z.B.
mittels
\color{blue}
\begin{lstlisting}
Sag mir, wieviele Nudeln auf Teller 1 liegen!
\end{lstlisting}
\color{black}
abfragen. Weitere zul\"assige Fragen in P(asta) resultieren in 
\color{purple}
\lstinline{Ja}
\color{black}
order
\color{purple}
\lstinline{Nein}
\color{black}
als m\"oglich Antworten.
So kann man z.B. mithilfe der Anweisung
\color{blue}
\begin{lstlisting}
Sag mir, ob Teller 1 voller als Teller 2 ist!
\end{lstlisting}
\color{black}
erfahren, ob auf Teller 1 mehr Nudeln liegen als auf Teller 2.
Alternativ, k\"onne man mittels
\color{blue}
\begin{lstlisting}
Sag mir, ob Teller 1 genauso voll wie Teller 2 ist!
\end{lstlisting}
\color{black}
erfragen, ob auf beiden Tellern die gleiche Anzahl von Nudeln liegt.

Weitere m\"ogliche Bedingungen sind
\color{blue}
\begin{lstlisting}
... Teller 1 leerer als Teller 2 ...
... Teller 1 nicht leerer als Teller 2 ...
... Teller 1 nicht voller als Teller 2 ...
... Teller 1 nicht genauso voll wie Teller 2 ...
... Teller 1 leer...
... Teller 1 nicht leer...
\end{lstlisting}
\color{black}
wobei die zu vergleichenden Teller nat\"urlich beliebige
innerhalb des allozierten Indexbereichs liegende Nummern
haben k\"onnen. Desweiteren k\"onnen Bedingungen durch 
\textcolor{red}{\lstinline{und}} bzw. \textcolor{red}{\lstinline{oder}}
verkn\"upft werden, z.B. w\"are
\color{blue}
\begin{lstlisting}
Teller 1 voller als Teller 2 
oder 
Teller 1 nicht genauso voll wie Teller 2
\end{lstlisting}
\color{black}
\"aquivalent zu 
\color{blue}
\begin{lstlisting}
Teller 1 nicht leerer als Teller 2
\end{lstlisting}
\color{black}

\paragraph{Dasselbe in C++}
Eine Aufforderung zur Eingabe in \Pasta\ kann in
C++ unter Verwendung von
\begin{lstlisting}
#include<iostream>
using namespace std;
\end{lstlisting}
z.B. wie folgt implementiert werden.
\begin{lstlisting}
cout << "Wieviele Nudeln liegen auf Teller 1?" << endl;
cin >> t[0];
\end{lstlisting}
Zur Ausgabe der Anzahl von Nudeln auf Teller 1 schreibt
man z.B.
\begin{lstlisting}
cout << "Auf Teller 1 ";
if (t[0]==1)
  cout << "liegt 1 Nudel." 
else
  cout << "liegen " << t[0] << " Nudeln." << endl;
\end{lstlisting}
Die Ausgabe von Wahrheitswerten kann z.B. wie folgt geschehen.
\begin{lstlisting}
bool c=t[0]==t[1];
// bool c=t[0]<t[1];
// bool c=!(t[0]<t[1]);
// bool c=!(t[0]>t[1]);
// bool c=t[0]!=t[1];
// bool c=t[0]==0;
// bool c=t[0]>0;
if (c)
   cout << "Ja." << endl;
else
   cout << "Nein." << endl;
\end{lstlisting}

\section{Optionale Anweisungen}

Optionale Anweisungen, deren Ausf\"uhrung vom Wahrheitswert einer Bedingung
abh\"angt, werden mittels eines \textcolor{red}{\lstinline{[Mache folgendes ... wenn]}}-Sprach-konstrukts implementiert: 
\color{red}
\begin{lstlisting}
Mache folgendes,
  <anweisung>,
wenn <Bedingung>!
\end{lstlisting}
\color{black}
Dabei wird die optionale Anweisung klein geschrieben und mit einem Komma
abgeschlossen.
So wird z.B. aufgrund
\color{blue}
\begin{lstlisting}
Mache folgendes,
  lege 1 Nudel auf Teller 2,
wenn Teller 1 voller als Teller 2 ist!
\end{lstlisting}
\color{black}
nur dann eine zus\"atzliche Nudel auf Teller 1 gelegt, wenn auf Teller 1 bereits
mehr Nudeln als auf Teller 2 liegen.

\paragraph{Dasselbe in C++}
Bedingte Folgen von Anweisungen k\"onnen in C++ in mittels
m\"oglicherweise geschachtelter \lstinline{if}-Anweisungen 
implementiert werden, z.B.
\begin{lstlisting}
if (c) {
  ...
}
\end{lstlisting}
wobei \lstinline{c} einen entsprechenden Wahrheitswert
bezeichnet.

\section{Schleifen}

Soll eine Anweisungen sooft wiederholt werden bis eine zu spezifizierende Bedingung
nicht mehr erf\"ullt ist, dann bietet sich die Verwendung des
\textcolor{red}{\lstinline{[Wiederhole ... solange]}}-Sprachkonstrukts an: 
\color{red}
\begin{lstlisting}
Wiederhole folgendes,
  <anweisung>,
solange <Bedingung>!
\end{lstlisting}
\color{black}
Wie bei optionalen Anweisungen wird die potentiell zu wiederholende Anweisung 
klein geschrieben und mit einem Komma abgeschlossen.
So wird z.B. aufgrund
\color{blue}
\begin{lstlisting}
Wiederhole folgendes,
  lege 1 Nudel auf Teller 2,
solange Teller 1 voller als Teller 2 ist!
\end{lstlisting}
\color{black}
wiederholt eine zus\"atzliche Nudel auf Teller 2 gelegt, bis auf Teller 2 
genauso viele
Nudeln wie auf Teller 1 liegen.

\paragraph{Dasselbe in C++}
\Pasta-Schleifen k\"onnen in C++ wie folgt imlementiert werden
\begin{lstlisting}
do {
  ...
} while (c);
\end{lstlisting}
wobei \lstinline{c} einen entsprechenden Wahrheitswert
bezeichnet.

\section{Anweisungssequenzen}

Sequenzen von Anweisungen werden durch einfache Aneinanderreihung von Anweisungen
gebildet:
\color{red}
\begin{lstlisting}
<Anweisung>!
<Anweisung>!
<Anweisung>!
\end{lstlisting}
\color{black}
Z.B. wird man mittels
\color{blue}
\begin{lstlisting}
Frag mich,
  wieviele Nudeln auf Teller 1 liegen sollen!
Sag mir,
  wieviele Nudeln auf Teller 1 liegen!
\end{lstlisting}
\color{black}
erst zur Eingabe eines Wertes aufgefordert, welcher dann
umgehend wieder ausgegeben wird.

Innerhalb von optionalen Anweisungen (\textcolor{red}{\lstinline{[Mache folgendes ... wenn]}}) und Schleifen
(\textcolor{red}{\lstinline{[Wiederhole folgendes ... solange]}})
werden einzelne Anweisungen klein geschrieben, mit Komma abgeschlossen und mittels 
\textcolor{red}{\lstinline{und}} getrennt:
\color{red}
\begin{lstlisting}
...
  <anweisung>, und
  <anweisung>, und
  <anweisung>,
...
\end{lstlisting}
\color{black}
Im folgenden Beispiel wird ein Wert solange abgefragt und unmittelbar wieder
ausgegeben, bis eine Null eingegeben wird.
\color{blue}
\begin{lstlisting}
Lege 1 Nudel auf Teller 1!
Wiederhole folgendes,
  frag mich,
    wieviele Nudeln auf Teller 1 liegen sollen,
  und
  sag mir,
    wieviele Nudeln auf Teller 1 liegen,
solange Teller 1 nicht leer ist!
\end{lstlisting}
\color{black}

\chapter{Beispiele}

\section{Grundrechenarten}

\subsection{Addition}

Zur Addition zweier nat\"urlicher Zahlen werden die beiden Summanden
durch Nudeln auf den Tellern 1 und 2 repr\"asentiert. Entsprechende
Zahlen von Nudeln werden dann jeweils auf Teller 3 gelegt, was das 
gesuchte Resultat ergibt.

\subsubsection{\Pasta\ Program}
\color{blue}
\lstinputlisting{../examples/addition.p}
\color{black}

\subsubsection{Beispielsitzung}
\color{purple}
\begin{lstlisting}
Wieviele Nudeln liegen auf Teller 1?
:-) 47
:
Wieviele Nudeln liegen auf Teller 2?
:-) 34
:
Auf Teller 3 liegen 81 Nudeln.
:
Guten Appetit!
\end{lstlisting}
\color{black}

\subsubsection{Dasselbe in C++}

\lstinputlisting{../../C++/addition.cpp}
\subsection{Subtraktion}

Zur Subtraktion zweier nat\"urlicher Zahlen wird der Minuend
durch Nudeln auf den Teller 1 und der Subtrahend entsprechend durch 
Nudeln auf Teller 2 repr\"asentiert. Dabei muss sichergestellt werden,
dass der Subtrahend nicht gr\"o\ss er als der Minuend ist. Sollte das
nicht der Fall sein, so wird der Wahrheitswert der Bedingung
\color{blue}
\begin{lstlisting}
Teller 2 voller als Teller 1
\end{lstlisting}
\color{black}
also \textcolor{purple}{\lstinline{Ja}} auf den Bildschirm ausgegeben.
Anderenfalls werden soviele Nudeln, wie auf Teller 2 liegen, von Teller
1 entfernt. Die auf Teller 1 verbleibende Anzahl von Nudeln stellt
das Resultat der Subtraktion dar.

\subsubsection{\Pasta\ Program}
\color{blue}
\lstinputlisting{../examples/subtraktion.p}
\color{black}

\subsubsection{Beispielsitzung}
\color{purple}
\begin{lstlisting}
Wieviele Nudeln liegen auf Teller 1?
:-) 47
:
Wieviele Nudeln liegen auf Teller 2?
:-) 34
:
Nein.
:
Auf Teller 3 liegen 13 Nudeln.
:
Guten Appetit!
\end{lstlisting}
\color{black}

\subsubsection{Dasselbe in C++}

\lstinputlisting{../../C++/subtraction.cpp}

\subsection{Multiplikation}

Zur Multiplikation zweier nat\"urlicher Zahlen werden die beiden 
Faktoren
durch Nudeln auf den Tellern 1 und 2 repr\"asentiert. W\"ahrend
einer Anzahl von Schleifeniterationen, die gleich der auf Teller 2
liegenden Zahl von Nudeln ist, werden jeweils soviele Nudeln auf Teller
3 gelegt, wie sich auf Teller 1 befinden. Das Resultat ergibt sich
anschlie\ss end zur Anzahl der auf Teller 3 liegenden Nudeln.

\subsubsection{\Pasta\ Program}
\color{blue}
\lstinputlisting{../examples/multiplikation.p}
\color{black}

\subsubsection{Beispielsitzung}
\color{purple}
\begin{lstlisting}
Wieviele Nudeln liegen auf Teller 1?
:-) 47      
:
Wieviele Nudeln liegen auf Teller 2?
:-) 34
:
Auf Teller 3 liegen 1598 Nudeln.
:
Gute Appetit!
\end{lstlisting}
\color{black}

\subsubsection{Dasselbe in C++}

\lstinputlisting{../../C++/multiplication.cpp}

\subsection{Division}

Zur ganzzahligen Division zweier nat\"urlicher Zahlen wird der Dividend
durch Nudeln auf den Teller 1 und der Divisor entsprechend durch 
Nudeln auf Teller 2 repr\"asentiert. Zur Berechnung des Quotienten und 
des Rests werden innerhalb einer Schleife jeweils soviele Nudeln, wie
auf Teller 2 liegen, von Teller 1 entfernt. Pro Schleifeniteration wird
der Wert des Quotienten auf Teller 3 um eins erh\"oht. Die Schleife
wird verlassen, sobald auf Teller 1 weniger Nudeln liegen als auf
Teller 2. Die auf Teller 1 verbleibenden Nudeln stellen den
Rest der ganzzahligen Division dar.

\subsubsection{\Pasta\ Program}
\color{blue}
\lstinputlisting{../examples/division.p}
\color{black}

\subsubsection{Beispielsitzung}
\color{purple}
\begin{lstlisting}
Wieviele Nudeln liegen auf Teller 1?
:-) 47
:
Wieviele Nudeln liegen auf Teller 2?
:-) 34
:
Auf Teller 3 liegt 1 Nudel.
:
Auf Teller 1 liegen 13 Nudeln.
:
Guten Appetit!
\end{lstlisting}
\color{black}

\subsubsection{Dasselbe in C++}

\lstinputlisting{../../C++/division.cpp}

\section{Was schon der kleine Gauss machen musste ...}

Vom Sch\"uler Carl-Friedrich Gauss wird sich folgende Geschichte
erz\"ahlt: 

``Eines Tages hatte sein Mathelehrer mal keine Lust und 
wollte sich durch eine seiner Meinung nach besonders aufw\"andige 
Aufgabe etwas Freizeit beschaffen. Er lie\ss\ die Klasse alle Zahlen 
von 1 bis 100 aufsummieren, in der Hoffnung, dass diese damit wenigstens
15 Minuten besch\"aftigt sei. 

Anstatt wild darauf los zu addieren, schaltete der kleine Gauss seinen
schon damals hellen Kopf ein und erkannte, dass sowohl 1+100 als auch
2+99 als auch 3+98 ... als auch 50+51 jeweils 101 ergaben. 
Er z\"ahlte insgesamt 50 dieser Summen und meldete nach wenigen Sekunden 
mit dem Resultat 5050=50*101 zur\"uck. Die erhoffte Pause des 
Mathelehrers wurde siginfikant verk\"urzt...''

Dank \Pasta\ m\"ussen wir nicht ganz so schlau wie Gauss sein.
Der Computer ist bekannterma\ss en bei der Summation gro\ss er
Zahlenfolgen signifikant effizienter als wir. \"Ubergeben wir 
also ihm den Job...

\subsubsection{\Pasta\ Program}
\color{blue}
\lstinputlisting{../examples/gauss.p}
\color{black}

\subsubsection{Beispielsitzung}
\color{purple}
\begin{lstlisting}
Wieviele Nudeln liegen auf Teller 1?
:-) 10
:
Auf Teller 3 liegen 55 Nudeln.
:
Guten Appetit!
\end{lstlisting}
\color{black}

\subsubsection{Dasselbe in C++}

\lstinputlisting{../../C++/gauss.cpp}

\section{Fakult\"at}

Die Berechnung der Fakult\"at einer nat\"urlichen Zahl
\"ahnelt dem Problem des kleinen Gauss bis auf die Tatsache,
dass die Elemente der Zahlenfolge nicht addiert sondern multipliziert
werden. Desweiteren ist per Definition $0!=1$, was durch die 
Initialisierung des Resultatstellers (Teller 3) mit einer Nudel
gew\"ahrleistet ist. Anschlie\ss end werden die ben\"otigten
Multiplikationen durchgef\"uhrt und das Resultat ausgegeben.

\subsubsection{\Pasta\ Programm}
\color{blue}
\lstinputlisting{../examples/fakultaet.p}
\color{black}
\subsubsection{Beispielsitzung}
\color{purple}

\begin{lstlisting}
Wieviele Nudeln liegen auf Teller 1?
:-) 5
:
Auf Teller 3 liegen 120 Nudeln.
:
Guten Appetit!
\end{lstlisting}
\color{black}

\subsubsection{Dasselbe in C++}

\lstinputlisting{../../C++/fakultaet.cpp}

\end{document}


\subsection{Beispiele}

\subsubsection{\tt gauss.p}

\begin{center}
\includegraphics[width=\textwidth]{figures/gauss.eps}
\end{center}

\section{Interpreter}

\chapter{Dasselbe in C}

Existierende Programmiersprachen 
basieren meist auf dem Englischen und sind sehr abstrakt gehalten, was
den Zugang oft stark erschwert bzw. unm\"oglich erscheinen l\"asst. 
Zum Beispiel ist 
\begin{lstlisting}
Hole Dir 2 Teller!
Lege 2 Nudeln auf Teller 1!
Lege 3 Nudeln auf Teller 2!
\end{lstlisting}
sicher intuitiver als
\begin{lstlisting}
int t1,t2;
t1+=2;
t2+=3;
\end{lstlisting}
Desweiteren erleichtert das Verst\"andnis eines \Pasta\ Programms 
den Zugang
zur abstrakteren Notation vieler ``richtiger'' Programmiersprachen 
wie z.B. C.

\paragraph{Nudeln sind ein Datentyp.} 

Nat\"urliche Zahlen $0,1,2,\ldots$ werden durch Nudeln bzw. deren Anzahl
repr\"asentiert. C bietet den Datentyp
``vorzeichenlose ganze Zahl"\ ({\tt unsigned int}) an.

\paragraph{Teller sind Variablen.}
Variablen haben einen Typ. In unserem Falle enthalten Teller Nudeln, was
Variablen von Typ ``vorzeichenlose ganze Zahl"\ entspricht. Durch
\begin{lstlisting}
Hole Dir 2 Teller!
\end{lstlisting}
 werden 2 Variablen dieses
Typs deklariert, auf die dann mittel \lstinline{Teller 1} und
\lstinline{Teller 2} zugegriffen werden kann. Die Bereitstellung
anderer Anzahlen von Tellern und der Zugriff auf diese funktionieren
analog. Neue Teller sind leer.
In C w\"urde man
\begin{lstlisting}
unsigned int t1, t2;
\end{lstlisting}
schreiben.

\paragraph{Zuweisungen von Werten an Variablen.}
Der Wert von Variablen kann durch Zuweisungen ver\"andert werden. 
Zum Beispiel
w\"urde man in C durch 
\begin{lstlisting}
t1=2; 
\end{lstlisting}
der Variable \lstinline{t1} den Wert 
\lstinline{2} zuweisen. In Anlehnung daran legen wir Nudeln auf Teller bzw.
entfernen diese wieder. \"Uberschreibung von Variablen erlauben wir nicht,
da dieses Konzept nicht intuitiv genug erscheint. Nehmen wir z.B. an, dass
Teller 1 eine Nudel enth\"alt. Eine Zuweisung von zwei Nudeln zu Teller 1
ist \"aquivalent zum Hinzuf\"ugen einer Nudel.
Die Erkl\"arung zur Zuweisung \lstinline{t1=2;} m\"usste demnach in etwa
wie folgt lauten: {\em Wir entfernen alle Nudeln von Teller 1 und legen
dann zwei drauf.} Ihr Kind wird Sie wahrscheinlich berechtigterweise fragen:
{\em Warum legen wir nicht einfach noch eine Nudel auf Teller 1?} ...
Genau das tun wir: 
\begin{lstlisting}
Lege 1 Nudel auf Teller 1!
\end{lstlisting}
Die Motivation der Umsetzung einer Zuweisung von
eins zu einer Variable mit dem Wert zwei mittels Wegnehmen funktioniert
analog.
\begin{lstlisting}
Nimm 1 Nudel von Teller 1!
\end{lstlisting}

\paragraph{Ausgaben.}
Die Anweisung
\begin{lstlisting}
Sag mir, wieviele Nudeln auf Teller 1 liegen!
\end{lstlisting}
bedingt die Ausgabe der entsprechenden Anzahl\footnote{Es m\"usste jetzt genau
eine Nudel draufliegen.} auf den Bildschirm.
Teller 2, Teller 3, usw. funktionieren analog.
In C w\"urde man 
\begin{lstlisting}
printf("%d\n",t1);
\end{lstlisting}
schreiben. Die Erkl\"arung dieses Ausdrucks ist Grundsch\"ulern
nicht zumutbar.


\paragraph{Gerechnet wird auch.}
Die Zahl der Nudeln, welche auf einen Teller gelegt bzw. von diesem
entfernt werden sollen, kann das Ergebnis einer beliebig komplexen Berechnung
sein. Dem Entwicklungsstand von Grundsch\"ulern entsprechend bieten wir
Addition und Subtraktion mit Werten von Variablen (Anzahl von Nudeln auf 
Tellern) und Konstanten (nat\"urlichen Zahlen) an. Nehmen wir z.B. an, dass
auf Teller 1 zwei und auf Teller 2 vier Nudeln liegen und dass auf
die Summe beider (sechs) zu Teller 1 hinzugf\"ugt werden soll.
\begin{lstlisting}
Lege 
  soviele Nudeln wie auf Teller 1 liegen 
  + 
  soviele Nudeln wie auf Teller 2 liegen 
auf Teller 1!
\end{lstlisting}
In C w\"urde man 
\begin{lstlisting}
t1+=t1+t2;
\end{lstlisting}
schreiben, was ohne Zweifel kompakter ist.\footnote{Unser Ziel ist nicht die 
Konvertierung von C-Programmierern in Anh\"anger unserer Sprache... 
Mit einiger Sicherheit erleichtert die Lekt\"ure dieses Buches jedoch die 
umgekehrte Entwicklung.}

\paragraph{Manchmal will man nur Ja oder Nein als Antwort.}

[2DO]

\paragraph{Bedingte Anweisungen.}
Manche Anweisungen sollen nur unter bestimmten Bedingungen ausgef\"uhrt
werden. Zum Beispiel will man nur dann eine Nudeln von 
Teller 1 nehmen, wenn dieser nicht leer ist. In C k\"onnte man
\begin{lstlisting}
if (t1!=0) t1-=1;
\end{lstlisting}
schreiben, was Ihr Kind wahrscheinlich nicht m\"ogen w\"urde...
In unserer Sprache darf
\begin{lstlisting}
Mach folgendes, 
  nimm 1 Nudel von Teller 1,
wenn
  Teller 1 nicht leer 
ist! 
\end{lstlisting}
geschrieben werden.

\paragraph{Eingaben zur Laufzeit des Programms.}

[2DO]

\paragraph{Schleifen.}

[2DO]


\bibliographystyle{plain}
\bibliography{book}

\appendix

\chapter{\Pasta\ Programmsammlung}

\section{Gerade und Ungerade Zahlen}

Jede gerade Zahl ist durch eine Summe von Zweien
$2+2+\ldots+2$ darstellbar. 

\subsubsection{\Pasta\ Programm}
\color{blue}
\lstinputlisting{../examples/beispiel010.p}
\color{black}

\subsubsection{Beispielsitzung}
\color{purple}
\begin{lstlisting}
Wieviele Nudeln liegen auf Teller 2?
:-) 3
:
Nein.
:
Guten Appetit!
\end{lstlisting}
bzw.
\begin{lstlisting}
Wieviele Nudeln liegen auf Teller 2?
:-) 22222
:
Ja.
:
Guten Appetit!
\end{lstlisting}
\color{black}

\section{Primzahlen}

Primzahlen sind nur als Summe von Einsen
$1+1+\ldots+1$ darstellbar. 

\subsection{Primzahltest}

\subsubsection{\Pasta\ Programm}
\color{blue}
\lstinputlisting{../examples/beispiel011.p}
\color{black}
\subsubsection{Beispielsitzung}
\color{purple}
\begin{lstlisting}
Wieviele Nudeln liegen auf Teller 1?
:-) 5
:
Ja.
:
Guten Appetit!
\end{lstlisting}
bzw.
\begin{lstlisting}
Wieviele Nudeln liegen auf Teller 1?
:-) 77
:
Nein.
:
Guten Appetit!
\end{lstlisting}
\color{black}

\subsection{Alle Primzahlen}

\subsubsection{\Pasta\ Program}
\color{blue}
\lstinputlisting{../examples/beispiel012.p}
\color{black}
\subsubsection{Beispielsitzung}
\color{purple}
\begin{lstlisting}
Wieviele Nudeln liegen auf Teller 6?
:-) 20
:
Auf Teller 1 liegen 2 Nudeln.
:
Auf Teller 1 liegen 3 Nudeln.
:
Auf Teller 1 liegen 5 Nudeln.
:
Auf Teller 1 liegen 7 Nudeln.
:
Auf Teller 1 liegen 11 Nudeln.
:
Auf Teller 1 liegen 13 Nudeln.
:
Auf Teller 1 liegen 17 Nudeln.
:
Auf Teller 1 liegen 19 Nudeln.
:
Guten Appetit!
\end{lstlisting}
\color{black}


\section{Fibonacci Zahlen}

\subsubsection{\Pasta\ Program}
\color{blue}
\lstinputlisting{../examples/fibonacci.p}
\color{black}

\subsubsection{Beispielsitzung}
\color{purple}

\begin{lstlisting}
Wieviele Nudeln liegen auf Teller 1?
:-) 8
:
Auf Teller 4 liegt 1 Nudel.
:
Auf Teller 4 liegen 2 Nudeln.
:
Auf Teller 4 liegen 3 Nudeln.
:
Auf Teller 4 liegen 5 Nudeln.
:
Auf Teller 4 liegen 8 Nudeln.
:
Auf Teller 4 liegen 13 Nudeln.
:
Auf Teller 4 liegen 21 Nudeln.
:
Auf Teller 4 liegen 34 Nudeln.
:
Guten Appetit!
\end{lstlisting}
\color{black}


\end{document}
