 \documentclass[12pt,twoside,a4paper]{scrartcl}

\usepackage[utf8]{inputenc}
\usepackage{amsmath}

\usepackage{wrapfig}
\usepackage{caption}
\usepackage{tcolorbox}
\usepackage{tabulary}
\usepackage{cite}


\author{Lennart Wilde}
\title{The Doctrine of M.A.D}

\begin{document}
    \maketitle
    \newpage

    \pagenumbering{arabic}
    \section{M.A.D}
    \subsection{What is Mutually Assured Destruction?}
        
        Mutually Assured Destruction (hereafter referred to as MAD) is the concept of the guaranteed destruction of any county to firstly use nuclear weapons in a conflict. MAD therefore ensures that nobody who starts a nuclear war can win - or even just survive it.
        
      \subsection{ Problems}
        \begin{center}
            \begin{itemize}
                \item \textbf{Guaranteed retaliation in the event of a first strike:} \\
                        Even if a first strike destroys major parts of the infrastructure the attacked country still has to be able to retaliate. This leads to "fail-deadly" systems which can easily escalate a local crisis to an all-out nuclear war. 
                
                \item \textbf{Balance:} \\
                    The whole concept relies on an equal vulnerability to nuclear weapons. If one side of a potential nuclear conflict has the ability to protect itself from a second strike or compromise nuclear weapons delivery, this delicate balance is overthrown, because a potential agressor does not have to fear annihilation.
                \item \textbf{Detection:} \\
                        The usage of a nuclear weapon has the be doubtlessly identifieable, otherwise a non nuclear agression might cause nuclear retaliation. Also the opposite is true: If a nuclear attack cannot be identified as such, the agressor has not to fear retaliation by nuclear weapons.
                        
                \item \textbf{Rationality:} \\
                        M.A.D only works if you assume that leaders act rationally. If the order to strike with a nuclear weapon is given out of a knee-jerk reaction without the consideration of the consequences, MAD does nothing to prevent it. A country already devestated by war it is loosing would also have a lower threshhold for a nuclear attack, as an last "all-out" attempt to turn the war. (Example: Nazi Germany and Kamikaze Attacks)
                \item \textbf{Non-State-Actors:}\\
                        Because of the lack of a defined territory a retaliation strike against them doesn't have the same impact as against a "regular" state.
                        
                \item \textbf{Attribution:} It can sometime be very difficult to determine the origin of an attack. If a nuclear retaliation is launched against the wrong country, it might have devestating effects.
            \end{itemize}
        \end{center}
        
        \subsection{Solutions}
            \begin{center}
            \begin{itemize}
                \item \textbf{Nuclear Verification}\\
                    Each country holding nuclear weapons has incentives to lie about the true numbers of their stockpile. If reliable numbers can be independently confirmed, the speculation of the surviveability of the retaliation strike can be eliminated. Also countrys like Israel, who deny the possession of nuclear weapons can be forced to deal with the responsibility.
                    
                \item \textbf{Nuclear Disarmament}\\
                    The ultimate goal to prevent a nuclear war, is to eliminate nuclear weapons. Unfortunately the classical prisoners dilemma applies here. Everyone would be better off if noone had nuclear weapons, but the mere threat of possesion forces other countries into the development of nuclear weapons.
            \end{itemize}
        \end{center}
\end{document}
